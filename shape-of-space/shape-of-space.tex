\documentclass[10pt,landscape]{article}
\usepackage{multicol}
\usepackage{calc}
\usepackage{ifthen}
\usepackage[landscape]{geometry}
\usepackage{hyperref}
\usepackage{amsmath}
\usepackage{esdiff}
\usepackage{cancel}
\usepackage{mathtools}

% To make this come out properly in landscape mode, do one of the following
% 1.
%  pdflatex latexsheet.tex
%
% 2.
%  latex latexsheet.tex
%  dvips -P pdf  -t landscape latexsheet.dvi
%  ps2pdf latexsheet.ps


% If you're reading this, be prepared for confusion.  Making this was
% a learning experience for me, and it shows.  Much of the placement
% was hacked in; if you make it better, let me know...


% 2008-04
% Changed page margin code to use the geometry package. Also added code for
% conditional page margins, depending on paper size. Thanks to Uwe Ziegenhagen
% for the suggestions.

% 2006-08
% Made changes based on suggestions from Gene Cooperman. <gene at ccs.neu.edu>


% To Do:
% \listoffigures \listoftables
% \setcounter{secnumdepth}{0}


% This sets page margins to .5 inch if using letter paper, and to 1cm
% if using A4 paper. (This probably isn't strictly necessary.)
% If using another size paper, use default 1cm margins.
\ifthenelse{\lengthtest { \paperwidth = 11in}}
	{ \geometry{top=.5in,left=.5in,right=.5in,bottom=.5in} }
	{\ifthenelse{ \lengthtest{ \paperwidth = 297mm}}
		{\geometry{top=1cm,left=1cm,right=1cm,bottom=1cm} }
		{\geometry{top=1cm,left=1cm,right=1cm,bottom=1cm} }
	}

% Turn off header and footer
\pagestyle{empty}
 

% Redefine section commands to use less space
\makeatletter
\renewcommand{\section}{\@startsection{section}{1}{0mm}%
                                {-1ex plus -.5ex minus -.2ex}%
                                {0.5ex plus .2ex}%x
                                {\normalfont\large\bfseries}}
\renewcommand{\subsection}{\@startsection{subsection}{2}{0mm}%
                                {-1explus -.5ex minus -.2ex}%
                                {0.5ex plus .2ex}%
                                {\normalfont\normalsize\bfseries}}
\renewcommand{\subsubsection}{\@startsection{subsubsection}{3}{0mm}%
                                {-1ex plus -.5ex minus -.2ex}%
                                {1ex plus .2ex}%
                                {\normalfont\small\bfseries}}
\makeatother

% Define BibTeX command
\def\BibTeX{{\rm B\kern-.05em{\sc i\kern-.025em b}\kern-.08em
    T\kern-.1667em\lower.7ex\hbox{E}\kern-.125emX}}

% Don't print section numbers
\setcounter{secnumdepth}{0}


\setlength{\parindent}{0pt}
\setlength{\parskip}{0pt plus 0.5ex}


% -----------------------------------------------------------------------

\begin{document}

\raggedright
\footnotesize
\begin{multicols}{3}


% multicol parameters
% These lengths are set only within the two main columns
%\setlength{\columnseprule}{0.25pt}
\setlength{\premulticols}{1pt}
\setlength{\postmulticols}{1pt}
\setlength{\multicolsep}{1pt}
\setlength{\columnsep}{2pt}

\begin{center}
  \Large{\textbf{The Shape of Space}~\cite{weeks2001shape}} \\
\end{center}

\section{Topology vs Geometry}

{\bf Topology of a surface:} the aspect of a surface's nature that is unaffected
by deformation.\\
{\bf Geometry of a surface:} consists of the properties that {\bf DO} change
when the surface is deformed.\\
{\bf Geometrical properties:} curvature (most important), areas, distances,
angles...


\section{Intrinsic vs Extrinsic properties}

{\bf Intrinsic topology:} same intrinsic topology if inside the surface one
cannot tell them apart.\\
{\bf Extrinsic topology:} same extrinsic topology if one can be deformed within
a higher-dimensional space to look like the other.\\
{\bf Intrinsic geometry:} properties of the surface.\\
{\bf Extrinsic geometry:} only to be appreciated from higher dimensions.\\
{\bf Geodesic:} {\underline{intrinsically}} straight line.\\


\section{Local vs Global properties}

{\bf Local properties} are those observable within a small region of the
manifold.\\
{\bf Global properties} require consideration of the manifold as a whole.\\
{\bf Homogeneous manifold:} one whose local geometry is the same at
\underline{all} points.\\
{\bf \underline{Remark}:} most often used in ``\underline{local geometry}'' and
``\underline{global topology}''.\\
{\bf Examples:}
\begin{itemize}
\item a two-dimensional manifold (surface) is a space with local topology of a
  plane. All two-manifolds have the same local topology.
\item a three-dimensional manifold is a space with local topology of
  ``ordinary'' 3D space; they all have the same local topology.
\end{itemize}


\section{Close vs Open}

Intuitively, \underline{closed} means finite and \underline{open} means
infinite.
{\bf Remark 1: \underline{Edges}:} anything with edges is NOT EVEN a manifold
(manifold-with-boundary; terms closed and open imply manifold has NO edges).\\
{\bf Remark 2: \underline{Area}:} there are surfaces that are infinitely long,
yet only have a finite area (cusp).\\
--- By convention, a surface is classified as closed or open accordingly to its
distance across rather than its area (cusp $=$ open).


\section{Orientability}

Manifolds that do not contain orientation-reversing paths are called orientable.
\begin{center}
  \begin{tabular}{r|c|c}
    & {\bf orientable} & {\bf non-orientable} \\\hline
    {\bf curved local geometry} & sphere & projective plane \\\hline
    {\bf flat local geometry} & torus & klein-bottle \\\hline
  \end{tabular}
\end{center}







\bibliographystyle{ieeetr}
\bibliography{shape-of-space.bib}

\clearpage


\end{multicols}
\end{document}
