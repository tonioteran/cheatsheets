\documentclass[10pt,landscape]{article}
\usepackage{multicol}
\usepackage{calc}
\usepackage{ifthen}
\usepackage[landscape]{geometry}
\usepackage{hyperref}
\usepackage{amsmath}
\usepackage{esdiff}
\usepackage{cancel}
\usepackage{mathtools}

% To make this come out properly in landscape mode, do one of the following
% 1.
%  pdflatex latexsheet.tex
%
% 2.
%  latex latexsheet.tex
%  dvips -P pdf  -t landscape latexsheet.dvi
%  ps2pdf latexsheet.ps


% If you're reading this, be prepared for confusion.  Making this was
% a learning experience for me, and it shows.  Much of the placement
% was hacked in; if you make it better, let me know...


% 2008-04
% Changed page margin code to use the geometry package. Also added code for
% conditional page margins, depending on paper size. Thanks to Uwe Ziegenhagen
% for the suggestions.

% 2006-08
% Made changes based on suggestions from Gene Cooperman. <gene at ccs.neu.edu>


% To Do:
% \listoffigures \listoftables
% \setcounter{secnumdepth}{0}


% This sets page margins to .5 inch if using letter paper, and to 1cm
% if using A4 paper. (This probably isn't strictly necessary.)
% If using another size paper, use default 1cm margins.
\ifthenelse{\lengthtest { \paperwidth = 11in}}
	{ \geometry{top=.5in,left=.5in,right=.5in,bottom=.5in} }
	{\ifthenelse{ \lengthtest{ \paperwidth = 297mm}}
		{\geometry{top=1cm,left=1cm,right=1cm,bottom=1cm} }
		{\geometry{top=1cm,left=1cm,right=1cm,bottom=1cm} }
	}

% Turn off header and footer
\pagestyle{empty}
 

% Redefine section commands to use less space
\makeatletter
\renewcommand{\section}{\@startsection{section}{1}{0mm}%
                                {-1ex plus -.5ex minus -.2ex}%
                                {0.5ex plus .2ex}%x
                                {\normalfont\large\bfseries}}
\renewcommand{\subsection}{\@startsection{subsection}{2}{0mm}%
                                {-1explus -.5ex minus -.2ex}%
                                {0.5ex plus .2ex}%
                                {\normalfont\normalsize\bfseries}}
\renewcommand{\subsubsection}{\@startsection{subsubsection}{3}{0mm}%
                                {-1ex plus -.5ex minus -.2ex}%
                                {1ex plus .2ex}%
                                {\normalfont\small\bfseries}}
\makeatother

% Define BibTeX command
\def\BibTeX{{\rm B\kern-.05em{\sc i\kern-.025em b}\kern-.08em
    T\kern-.1667em\lower.7ex\hbox{E}\kern-.125emX}}

% Don't print section numbers
\setcounter{secnumdepth}{0}


\setlength{\parindent}{0pt}
\setlength{\parskip}{0pt plus 0.5ex}


% -----------------------------------------------------------------------

\begin{document}

\raggedright
\footnotesize
\begin{multicols}{3}


% multicol parameters
% These lengths are set only within the two main columns
%\setlength{\columnseprule}{0.25pt}
\setlength{\premulticols}{1pt}
\setlength{\postmulticols}{1pt}
\setlength{\multicolsep}{1pt}
\setlength{\columnsep}{2pt}

\begin{center}
     \Large{\textbf{Dynamics Cheat Sheet}} \\
\end{center}

\section{Linear Momentum $\mathbf{L}$}
    {\bf Definition:} total force acting on a particle is equal to the time
    rate of change of its linear momentum.\\
    {\bf Equation:} $\mathbf{L} = m\mathbf{v}$ \\
    - for a system of particles ($G$: center of mass):
    \[ \mathbf{L} = \sum_{i} m_i \mathbf{v}_i = m \mathbf{v}_G \]
    {\bf Time rate of change:} $\mathbf{\dot{L}} = \mathbf{F}$ \\
    - for a system of particles ($G$: center of mass):
    \[ \mathbf{\dot{L}} = \sum_{i} m_i \mathbf{a}_i = \sum_i \mathbf{F}_i = 
    \underbrace{\mathbf{F}}_{\text{external forces}} = m \mathbf{a}_G \]


\section{Angular Momentum $\mathbf{H}$}

    {\bf Definition:} "moment" of the particle's  linear momentum $\mathbf{L}$
    about a point $O$.\\

    {\bf Equation:} $\mathbf{H}_{O} = \mathbf{r} \times \mathbf{L} = \mathbf{r} 
    \times m\mathbf{v}$ \\
    {\bf Equation:} $\mathbf{H}_{O} = \mathbf{r}_G \times m\mathbf{v}_G + \mathbf{H}_G$ \\
    {\bf Equation:} $\mathbf{H}_{G} = [\mathbf{I}_G]\boldsymbol{\omega} $ \\
    - for a system of particles (about fixed point $O$):
    \begin{align*}
        \mathbf{H}_O &= \sum_i \left( \mathbf{r}_i \times m_i \mathbf{v}_i \right)
    \end{align*}
    {\bf Equation:} taken about the center of mass $G$:
    \begin{align}
        \mathbf{H}_G &= \sum_i \left( \mathbf{r}_i' \times m \mathbf{v}_i \right) \;\;\; | \;\; \mathbf{v}_i = 
            \mathbf{\dot{r}}_i = \mathbf{\dot{r}}_G + \mathbf{\dot{r}}_i' = \mathbf{v}_G + \mathbf{v}_i' \\
        &= \sum_i \left( \mathbf{r}_i' \times m \left( \mathbf{v}_G + \mathbf{v}_i' \right) \right) \nonumber\\
        &= \sum_i \left( \mathbf{r}_i' \times m \mathbf{v}_G \right) + 
           \sum_i \left( \mathbf{r}_i' \times m \mathbf{v}_i' \right)\;\; | 
           \; -\mathbf{v}_G \times \sum_i m_i \mathbf{r}_i' = 0  \nonumber\\
        \mathbf{H}_G &= \sum_i \left( \mathbf{r}_i' \times m \mathbf{v}_i' \right)
    \end{align}
    Equation (1) is the {\bf \em absolute angular momentum}, whereas
    equation (2) is the {\bf \em relative angular momentum}; when taken about 
    $G$, these quantities are identical.

    {\bf Time rate of change:} wrt fixed point $O$
    \begin{align*}
        \diff{\mathbf{H}_O}{t} &= \cancelto{0}{\mathbf{\dot{r}} \times m\mathbf{v}} + 
        \mathbf{r} \times m\mathbf{\dot{v}}\;\;\; | \;\; \mathbf{\dot{r}} = \mathbf{v}    \\
        &= \mathbf{r} \times m\mathbf{a}\;\;\; | \;\; \mathbf{F} = m\mathbf{a}    \\
        &= \mathbf{r} \times \mathbf{F} \\
        &= \mathbf{M}_O
    \end{align*}
    - for a system of particles (about fixed point $O$):
    \begin{align*}
        \mathbf{\dot{H}}_O &= \sum_i \left( \cancelto{0}{\mathbf{\dot{r}}_i \times m_i \mathbf{v}_i} \right) + 
            \sum_i \left( \mathbf{{r}}_i \times m_i \mathbf{a}_i \right) \\
        &= \sum_i \left( \mathbf{{r}}_i \times \mathbf{F}_i \right) + \sum_i M_i \;\;\; | \;\; M_i 
            = \text{external moments} \\
        &= \mathbf{M}_O
    \end{align*}
    {\bf Time rate of change:} about CoM $G$:
    \[\mathbf{\dot{H}}_G = \sum_i \left( \mathbf{r}_i' \times m \mathbf{\dot{v}}_i' \right) = 
        \sum_i \left( \mathbf{r}_i' \times \mathbf{F}_i \right) + \sum_i M_i = \mathbf{M}_G,\]
    where $\mathbf{M}_G$ is the total moment about $G$ of the applied external
    forces plus any external moments. Expression valid for any movement of $G$! \\
    {\bf Time rate of change:} 3D rigid body, body-fixed frame:
    \begin{align*}
        \mathbf{\dot{H}} &= [\mathbf{I}]\boldsymbol{\dot{\omega}} + \boldsymbol{\omega} \times \mathbf{H} = \mathbf{M} \\
        M_x &= \dot{H}_x - H_y\omega_z + H_z \omega_y \\
        M_y &= \dot{H}_y - H_z\omega_x + H_x \omega_z \\
        M_z &= \dot{H}_z - H_x\omega_y + H_y \omega_x 
    \end{align*}
    {\bf Remark:} internal forces and motions {\bf will not } change angular momentum!


\section{Conservation of Angular Momentum}
    Several ways to express, but the best choices are:\\
    (i) about the center of mass $G$:
    \[ I_G \boldsymbol{\alpha} = \mathbf{M}_G = \mathbf{\dot{H}}_G \]
    (ii) about fixed point $O$:
    \[ I_O \boldsymbol{\alpha} = \mathbf{M}_O \]


\section{Angular Impulse}
    \[ \int_{t_1}^{t_2}{\mathbf{M}_O}dt = \int_{t_1}^{t_2}{\mathbf{\dot{H}}_O}dt 
        = \mathbf{H}_O(t_2) - \mathbf{H}_O(t_1) = \Delta \mathbf{H}_O \]
    "Useful when dealing with impulsive forces. Possible to calculate the 
    integrated effect of a force on a particle without knowing in detail the 
    actual value of the force as a function of time."\\


\section{Kinetic Energy {T}}
    {\bf Equation:} $T = T_T + T_R$, translational and rotational parts\\
    {\bf Equation:} for a system of particles:
    \begin{align*}
        T &= \sum_i T_i = \sum_i \left( \frac{1}{2} m_i \mathbf{v}_i \cdot \mathbf{v}_i \right) \;\; | \;\;
            \text{relative and $G$ parts} \\
          &= \frac{1}{2} m v_G^2 + \sum_i \frac{1}{2} m_i {v_i'}^2
    \end{align*}
    {\bf Equation:} for a 3D rigid body:
    \begin{align*}
        T &= \frac{1}{2} m v_G^2 + \frac{1}{2} \boldsymbol{\omega}\cdot \mathbf{H}_G \\
          &= \underbrace{\frac{1}{2} \mathbf{v}_G \cdot \mathbf{L}_G}_{\text{translation}} + \underbrace{\frac{1}{2} \boldsymbol{\omega}\cdot \mathbf{H}_G}_{\text{rotation}}
    \end{align*}
    {\bf Equation:} fxd pt $O$ about which body rotates:
    \begin{align*}
        T &= \frac{1}{2} \boldsymbol{\omega}\cdot \mathbf{H}_O 
    \end{align*}
    {\bf Equation:} instantaneous center of rot $C$
    \begin{align*}
        T &= \frac{1}{2} \boldsymbol{\omega}\cdot \mathbf{H}_C 
    \end{align*}
    {\bf Equation:} if $\boldsymbol{\omega}$ expressed in principal axes
    \begin{align*}
        T &= \frac{1}{2} m v_G^2 + 
            \frac{1}{2} \left( I_{xx}\omega_x^2 + I_{yy}\omega_y^2 + I_{zz}\omega_z^2 \right)
    \end{align*}
    {\bf Remark:} can provide considerable insight into the behavior of a 
    rotating body.\\
    {\bf Remark:} no assurance that energy will be conserved; internal motions
    can dissipate energy.


\section{General Motion (Chasles' Theorem)}
    Velocity and acceleration of a point $P$ in a rigid body:
    \begin{align*}
        \mathbf{v}_P &= \mathbf{v}_{O'} + \boldsymbol{\omega} \times \mathbf{r}_P' \\
        \mathbf{a}_P &= \mathbf{a}_{O'} + \boldsymbol{\dot{\omega}} \times \mathbf{r}_P' + 
            \boldsymbol{{\omega}} \times (\boldsymbol{\omega} \times \mathbf{r}_P'),
    \end{align*}
    where $O'$ is a known point in the rigid body.


\section{Moments of Inertia}
    Mass distribution, always positive. $(\cdot)'$ relative to point
    \begin{align*}
        I_{xx} &= \int_m \left( {y'}^2 + {z'}^2 \right) dm \\
        I_{yy} &= \int_m \left( {x'}^2 + {z'}^2 \right) dm \\
        I_{zz} &= \int_m \left( {x'}^2 + {y'}^2 \right) dm
    \end{align*}


\section{Products of Inertia}
    Imbalances in the mass distribution; pos, neg, or 0. $(\cdot)'$ relative to point
    \begin{align*}
        I_{xy} &= I_{yx} = \int_m {x'}{y'} dm \\
        I_{xz} &= I_{zx} = \int_m {x'}{z'} dm \\
        I_{yz} &= I_{zy} = \int_m {y'}{z'} dm
    \end{align*}

\section{Tensor of Inertia}
    How mass is distributed in a rigid body. Symmetric matrix. Causes $\mathbf{H}$
    and $\boldsymbol{\omega}$ to {\bf not} necessarily be parallel!
    \[ [\mathbf{I}_G] = \begin{bmatrix} I_{xx} & -I_{xy} & -I_{xz} \\
                                        -I_{yx} & I_{yy} & -I_{yz} \\
                                        -I_{zx} & -I_{zy} & I_{zz} \end{bmatrix} \]


\section{Principal Axes of Inertia}
    Frame of reference for which products of inertia are always 0 (diagonal $[\mathbf{I}_G]$).
    \[\mathbf{H}_G = I_x \omega_x \mathbf{\hat{i}} + 
        I_y \omega_y \mathbf{\hat{j}} + I_z \omega_z \mathbf{\hat{k}} \]


\section{Parallel Axis Theorem}
    Get inertia tensor about different axes (about $O$). $(\cdot)'$ relative to $G$
    \begin{align*}
        (I_{xx})_O &= I_{xx} + m(y_G^2 + z_G^2) \\
        (I_{xy})_O &= I_{xy} + m x_G y_G
    \end{align*}

\section{*Principal Axes as Eigenvalue Problem}


\section{Center of Mass}
    \[\mathbf{r}_G = \frac{1}{m}\left(\sum_i m_i \mathbf{r}_i\right); m = \sum_i m_i\]


\section{Tangential Velocity}
\[\mathbf{v} = \boldsymbol{\omega} \times \mathbf{r}\]


\section{General Eqns for 3D Rotational Motion}
    The governing equations are those of conservation of:
    \begin{align*}
        \text{linear momentum } \mathbf{L} &= M\mathbf{v}_G, \;\mathbf{\dot{L}} = \mathbf{F}  \\
        \text{angular momentum } \mathbf{H} &= [\mathbf{I}]\boldsymbol{\omega}, 
            \;\mathbf{\dot{H}}_G = \mathbf{M}_G
    \end{align*}


\section{Euler's Equations}
    Body-fixed frame aligned with principal axes:
    \begin{align*}
        M_x &= I_{xx}\dot{\omega}_x - (I_{yy} - I_{zz} ) \omega_y\omega_z \\
        M_y &= I_{yy}\dot{\omega}_y - (I_{zz} - I_{xx} ) \omega_z\omega_x \\
        M_z &= I_{zz}\dot{\omega}_z - (I_{xx} - I_{yy} ) \omega_x\omega_y
    \end{align*}


\section{Coreolis Theorem}
    Time derivative of a vector in a rotating frame. Consider $W$ a fixed frame,
    $B$ a rotating frame with angular velocity $\boldsymbol{\Omega}$, and a vector
    $\mathbf{r}$:
    \[ \prescript{W}{}{ \mathbf{\dot{r}} } = \diff{\prescript{W}{}{ \mathbf{r} }}{t} =
       \prescript{B}{}{ \mathbf{\dot{r}} } + \boldsymbol{\Omega} \times \prescript{B}{}{ \mathbf{r} } \]


\section{*Space and Body Cones}
\section{*Conservation of Energy}


\clearpage


\end{multicols}
\end{document}
